%\begin{center}
%\large \bf \runtitulo
%\end{center}
%\vspace{1cm}
\chapter*{\runtitulo}

\noindent Es esperado que muchos sistemas corran continuamente mientras el ambiente cambia y los requerimientos
evolucionan, por lo tanto las implementaciones de dichos sistemas deben ser actualizados dinámicamente para satisfacer
los cambios de requerimientos, respetando los cambios del ambiente. Lo complejo de este paso, es poder determinar en que
puntos de la ejecución previa es seguro hacer la actualización, y si es seguro, como deberá seguir ejecutando el nuevo
sistema. A su vez, existe la necesidad de desarrollar técnicas que permitan actualizar un sistema sin frenar o
interrumpir la ejecución del sistema.

\noindent Tanto la máquina, como el ambiente y los requerimientos, pueden ser interpretados por modelos de
comportamiento que son estructuras formales que definen acciones que pueden suceder. Luego, con la técnica de síntesis de
controladores podremos obtener modelos de forma correcta debido a que son obtenidos mediante construcciones.

\noindent En esta tesis presentaremos una solución general que no solo produce un controlador para la nueva
especificación y maneja la transición de uno a otro, sino que también, fuerza al sistema a que alcance un estado en el
cual la transición puede ocurrir de manera segura.

\noindent Finalizando, desarrollaremos varios casos de estudio utilizando la herramienta MTSA (Modal Transition System
Analyser) que nos permite efectuar la síntesis de controladores. Los casos de estudios fueron tomados de trabajos
previos sobre actualización dinámica y sistemas adaptables lo que nos permite hacer un trabajo comparativo entre
nuestros resultados y los obtenidos previamente.

 
\bigskip

\noindent\textbf{Palabras claves:} \textit{Síntesis de controladores; LTS; Actualización dianámica; Sistemas Adaptables.}
