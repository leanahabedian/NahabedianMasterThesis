%\begin{center}
%\large \bf \runtitulo
%\end{center}
%\vspace{1cm}
\chapter*{\runtitulo}

\noindent Es esperado que muchos sistemas corran continuamente mientras el ambiente cambia y los
requerimientos evolucionan, por lo tanto las implementaciones de dichos sistemas deben ser
actualizados dinámicamente para satisfacer los cambios de requerimientos, respetando los cambios
del ambiente. Lo complejo de este paso, es poder determinar en que puntos de la ejecución previa
es seguro hacer la actualización, y si es seguro, como deberá seguir ejecutando el nuevo sistema.
Tanto la máquina, como el ambiente y los requerimientos, pueden ser interpretados por modelos de 
comportamiento que son estructuras formales que definen acciones que pueden suceder. Mediante la
síntesis de controladores podremos obtener modelos de forma correcta debido a que son obtenidos 
mediante construcciones. 

\noindent El enfoque de esta tesis es definir y plantear formalmente mediante síntesis de
controladores el problema de la actualización dinámica, detallando un conjunto de inputs 
necesarios para la solución del mismo. A su vez, desarrollaremos varios casos de test utilizando la
herramienta MTSA dejando constancia de que los inputs definidos son suficientes para obtener el
controlador buscado.

 
\bigskip

\noindent\textbf{Palabras claves:} \textit{Ingeniería de requerimientos; Síntesis de controladores;
Actualización dianámica; Especificación basada en eventos; MTSA framework; Concurrencia, LTS;
Fluent; LTL.}
