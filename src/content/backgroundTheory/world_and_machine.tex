\section{El Mundo y la Máquina}

Los primeros conceptos que detallaré estarán involucrados con la ingeniería de los requerimientos. Los puntos de vista
mas relevantes son los de Zave y Jackson (\cite{Zave97fourdark, Jackson:1995:SRA:210207, 5071113}) por un lado, y los de
Letier y Van Lamsweerde (\cite{879820, VanLamsweerde:2001:GRE:882477.883624}) por el otro. Ambos puntos de vista
distinguen a los problemas del \emph{Mundo} y las soluciones de la \emph{Máquina} como fundamentales para reconocer si los
objetivos del sistema están siendo alcanzados correctamente. De hecho, el efecto de la máquina en el mundo y las
suposiciones que hacemos acerca de este mundo son fundamentales para el proceso de toma de requerimientos. En el lado
del mundo definimos una serie de problemas que existen en el mundo real que serán solucionados al construir una máquina.
Fácilmente podremos notar que existen componente en la máquina que interactuan directamente con el mundo siguiendo
normas y procesos conocidos. Por ejemplo
