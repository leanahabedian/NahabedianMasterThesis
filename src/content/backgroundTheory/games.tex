\section{Juegos de dos jugadores}

Llamaremos juegos de dos jugadores a aquellos que consisten en dos jugadores, jugador 1 y jugador 2, donde el objetivo
del jugador 1 es satisfacer una especificación independientemente de las acciones que el jugador 2 ejecute.
Intuitivamente, el jugador 1 puede deshabilitar las acciones que él controla aunque no podrá deshabilitarlas todas ya
que esto transformaría dicho estado a un estado de deadlock.

Durante el transcurso de esta tesis llevaremos los juegos de dos jugadores al marco de síntesis de controladores, donde
el jugador 1 (el controlador) elije, del conjunto de acciones controlables, cual habilitar y el jugador 2 (el ambiente)
elije que acciones tomar libremente. Formalmente podemos definir un juego de dos jugadores de la siguiente manera:

\begin{nahaDef}
    \emph{(Juego de dos jugadores) Un juego de dos jugadores es $G = (S_g, \Gamma^-, \Gamma^+, s_{g_0}, \varphi)$, donde
    $S$ es un conjunto finito de estados, $\Gamma^-, \Gamma^+ \subseteq S \times S$ son conjuntos de transiciones no
    controlables y controlables respectivamente, $s_{g_0} \in S$ es el estado inicial, y $\varphi \subseteq S^\omega$ es
    la condición de ganada. Definimos $\Gamma^-(s) = \{s'\ |\ (s,s') \in \Gamma^-\}$ y análogamente para $\Gamma^+$. Un
    estado $s$ es no controlable si $\Gamma^-(s) \neq \emptyset$ y controlable en el resto de los casos. Una jugada en
    $G$ es una secuencia $p = s_{g_0}, s_{g_1},...$ Una jugada $p$ terminada en $s_{g_n}$ es extendida por el
    controlador eligiendo un subconjunto $\gamma \subseteq \Gamma^+(s_{g_n})$. Luego el ambiente elige un estado
    $s_{g_{n+1}} \in \gamma \cup \Gamma^-(s_{g_n})$ y agrega $s_{g_{n+1}}$ a $p$.}
\end{nahaDef}

Un detalle importante es que si para un estado controlable $\gamma$ el conjunto de opciones del controlador es vacía, esto
puede llevar a un deadlock. Esto será considerado como prohibido más adelante, ya que el controlador definirá este estado
como un estado perdedor. Para un estado no controlable el controlador puede decidir deshabilitar todas las acciones
controlables. Las elecciones del controlador son formalizadas como estrategias y estas reglas son las que el controlador
aplicará. Por lo general, las estrategias son elegidas dependiendo de la historia. Esto puede verse en la estrategia
utilizando un valor de memoria $\Omega$ y actualizando este valor de acuerdo a la evolución del juego.

Es importante destacar, que este tipo de juegos, con memoria, es diferente al definido en \cite{Piterman}. Piterman et al.
definen un juego en el cual el ambiente elige su movimiento y recién luego de este, el controlador podrá elegir cual será
el siguiente paso.

\begin{nahaDef}
    \emph{(Estrategia con memoria) Una estrategia con memoria $\Omega$ para el controlador es un par de funciones
    $(\sigma, u)$, donde $\Omega$ es una memoria que tiene designado como valor inicial $\omega_0, \sigma: \Omega \times
    S \rightarrow 2^S$ tal que $\sigma(\omega,s) \subseteq \Gamma^+(s)$ y $u: \Omega \times S \rightarrow \Omega$.}
\end{nahaDef}

Intuitivamente, $\sigma$ le informa al controlador cuales estados debe habilitar como posibles sucesores y $u$ define
como actualizar la memoria en cada paso. Si $\Omega$ es finita, diremos que la estrategia usa memoria finita.

\begin{nahaDef}
    \emph{(Consistencia y estrategia ganadora) una jugada finita o infinita $p = s_0,s_1,...$ es consistente con
    $(\omega,u)$ si para cada $n$ tenemos que $s_{n+1} \in \sigma(\omega_n,s_n)$ donde $\omega_{i+1} = u(\omega_i,s_{i+1})$ 
    para toda $i \geq 0$. Una estrategia $(\sigma, u)$ para el controlador desde el estado $s$ es
    ganadora si cada jugada maximal empezando de $s$ y consistente con $(\sigma, u)$ es infinita y en $\varphi$. Diremos
    que el controlador gana el juego $G$ si tiene una estrategia ganadora desde el estado inicial.}
\end{nahaDef}

Diremos que verificar si un controlador gana un juego $G$ es resolver el juego $G$. Una vez definido un juego de dos
jugadores, pasaremos a traducir un problema de síntesis de controladores a este tipo de juegos. La transformación se
basa en generar una estrategia ganadora para el controlador. Si dicha estrategia existe, diremos que el problema de
control es realizable \cite{MalerPS95,21072}. Resultados estudiados anteriormente \cite{Pnueli:1989:SRM:75277.75293},
demuestran que si un controlador gana el juego $G$ y $\varphi$ es $\omega$-regular, el juego puede ganarse utilizando
una estrategia con memoria finita.
