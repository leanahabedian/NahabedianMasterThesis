\section{Sistema de Transición Etiquetados (Labelled Transition System)}

En esta sección vamos a dar una notación para los sistemas de transiciones etiquetados o Labelled Transition System
(LTS), la cual usaremos durante este trabajo. Dichos sistemas, son muy usados actualmente para modelar y analizar
comportamiento en sistemas concurrentes y distribuidos. Un LTS es un sistema de transiciones de estados donde cada una
de ellas esta etiquetada con una acción. El conjunto de todas las acciones que posee un LTS es llamado alfabeto.

\begin{nahaDef}
    \emph{(Sistema de Transición Etiquetado)\cite{Keller:1976:FVP:360248.360251} Sea $States$ un conjunto universal de estados, $Act$ un conjunto
    universal de etiquetas. Un Sistema de Transición Etiquetado (LTS) es una tupla $E = (S_E,A_E,\Delta_E,s_{E_0})$,
    donde $S_E \subseteq States$ es un conjunto finito de estados, $A_E \subseteq Act$ es un alfabeto finito, $\Delta_E
    \subseteq (S_E \times A_E \times S_E)$ es una relación, y $s_0 \in S_E$ es el estado inicial.}
\end{nahaDef}

Si $(s,\ell,s') \in \Delta_E$ diremos que $\ell$ está activo desde $s$ en $E$. Diremos también que un LTS $E$ es
\emph{determinístico} si $\forall_{(s,\ell,s'),(s,\ell,s'') \in \Delta_E}$ implica $s' = s''$. Para un estado $s$ definiremos
$\Delta_E(s) = $\{$\ell$ $|$ $(s,\ell,s')$ $\in$ $\Delta_E\}$. Dado un LTS $E$, vamos a referirnos a su alfabeto como $\alpha E$.

\begin{nahaDef}
    \emph{(Composición en Paralelo) Sean $M = (S_M,A_M,\Delta_M, s_{M_0})$ y $E = (S_E,A_E,\Delta_E, s_{E_0})$ LTSs.
    Una Composición en Paralelo($||$) es un operador simétrico tal que $E||M$ es el LTS definido de la siguiente
    manera $E||M = (S_E \times S_M, A_E \cup A_M, \Delta, (s_{E_0},s_{M_0}))$, donde $\Delta$ es la relación mas
    pequeña que satisface las siguientes reglas, donde $\ell \in A_E \cup A_M$:}
\end{nahaDef}

ESCRIBIR GARCHA DE LOGICA

\begin{nahaDef}
    \emph{(LTS Legal) Dado $E = (S_E, A_E, \Delta_E, s_{E_0}), M = (S_M, A_M, \Delta_M, s_{M_0})$ LTSs, y $A_{E_u} \in
    A_E$. Decimos que $M$ es un LTS $Legal$ par $E$ con respecto a $A_{E_u}$ si para todos $(s_E,s_M) \in E||M$ sucede
    lo siguiente: $\Delta_{E||M}((s_E,s_M)) \cap A_{E_u} = \Delta_E(S_E) \cap A_{E_u}$.}
\end{nahaDef} 

Intuitivamente, un LTS $M$ es un LTS $Legal$ para el LTS $E$ con respecto a $A_{E_u}$, si para todos los estados en la
composición $(s_E,s_M) \in S_{S||M}$ se cumple que, una acción $\ell \in A_{E_u}$ es deshabilitada en $(s_E,s_M)$ si y solo
si también esta deshabilitada en $s_E \in E$. En otras palabras, $M$ no restringe a $E$ con respecto a $A_{E_u}$.


\begin{nahaDef}
    \emph{(Tranzas) Sea un LTS $E = (S,A,\Delta,s_0)$. Una secuencia $\pi = \ell_0,\ell_1,...$ es una traza en $E$ si existe una
    secuencia $s_0,\ell_0,s_1,\ell_1,...$ donde para todo $i$ tenemos $(s_i,\ell_i,s_{i+1}) \in \Delta$.}
\end{nahaDef} 


\begin{nahaDef}
    \emph{(Estados Alcanzables) Sea un LTS $E = (S_E, A_E, \Delta_E, s_0)$. Un estado $s \in S_E$ es alcanzable (desde
    el estado inicial) en $E$ si existe una secuencia $s_0,\ell_0,s_1,\ell_1,...$ donde para cada $i$ tenemos
    $(s_i,\ell_i,s_{i+1}) \in \Delta$ y $s = s_{i+1}$. Nos referimos a el conjunto de todos los estados alcanzables en $E$
    como $Reach(E)$.}
\end{nahaDef} 

En el transcurso de esta tesis, vamos a estudiar solo aquellos LTSs $E$ donde todos sus estados $s \in S_E$ son
alcanzables.
