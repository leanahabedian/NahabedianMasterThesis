\section{Procesos de estados Finitos (Finite State Process)}

A esta altura, ya hemos definido las LTSs definiendo sus componentes, como lo son, sus estados, sus acciones, sus
transiciones y su estado inicial. Esta representación es adecuada para LTSs con pocos estados, pero se vuelve muy poco
práctica a la hora de trabajar con LTSs de gran tamaño. Por esta razón, usamos una simple notación de álgebra de
procesos llamada procesos de estados finitos (FSP: Finite State Process) para especificar LTSs.
\cite{644733,Magee:2000:CSM:332036}

El FSP es un lenguaje de especificación de semántica bien definida en términos de LTSs que provee describirlos de
manera concisa. Cada expresión FSP $E$ puede ser relacionada a un LTS finito. Notaremos $lts(E)$ al LTS que corresponde a dicho FSP.
A continuación discutiremos detalladamente la sintaxis del FSP.

A modo de ejemplo, en la Figura \ref{FSP}, mostramos un código FSP que representa el funcionamiento de una planta nuclear.

En FSP, los nombres de los procesos empiezan con letras mayúsculas y las acciones con minúsculas. El código de la planta
nuclear consta de dos procesos FPS, el primero, llamado \texttt{\textbf{MAINTENANCE}} modela el proceso de enviar un
mensaje para que se realice el mantenimiento de la bomba refrigeradora y recibe la respuesta de dicho mensaje. Estas
acciones se representan con la acción \texttt{\textbf{request}} y \texttt{\textbf{ok}} respectivamente. Por otro lado,
tenemos el proceso \texttt{\textbf{COOLER}} que posee como procesos auxiliares a los subprocesos \texttt{\textbf{STARTED}}
y \texttt{\textbf{STOPPED}} que son locales al proceso FSP en donde están definidas. \texttt{\textbf{COOLER}} está
definida para que inicialmente se comporte como \texttt{\textbf{STARTED}} puesto que queremos modelar que la bomba en
estado inicial esta prendida. Luego, podemos ejecutar diferentes acciones, \texttt{\textbf{stopPump}},
\texttt{\textbf{procedure}} y \texttt{\textbf{ok}}. \texttt{\textbf{STARTED}} está definido usando el operador de acción 
\texttt{\textbf{->}} y recursión. Por ejemplo, dicho proceso está definido para empezar ejecutando, o bien \texttt{\textbf{procedure}}
o \texttt{\textbf{ok}}, acciones que nos llevan a seguir ejecutando como el proceso \texttt{\textbf{STARTED}} indica, o
\texttt{\textbf{stopPump}} que nos llevará a ejecutar el proceso \texttt{\textbf{STOPPED}}.

\begin{figure}
    \centering
    \includegraphics[scale=0.50]{img/FSP.png}
    \caption{Ejemplo FSP}
    \label{FSP}
\end{figure}

A su vez, los FSP soportan distintos operadores de composición como la composición en paralelo. Dicha operación,
denotada como $|\ |$, esta definida para preservar la semántica de la composición en paralelo de los LTS definidos en la
definición \ref{COMP_EN_PARALELO}. Por lo tanto, dados dos procesos FSP \texttt{P} y \texttt{Q}, tenemos:
\texttt{lts(P||Q) = lts(P)||lts(Q)}.

En procesos FSP que están definidos mediante una composición de dos procesos no auxiliares, son llamados procesos
compuestos y sus nombres poseen el prefijo $|\ |$. En nuestro ejemplo, la composición en paralelo entre los procesos FSP 
\texttt{\textbf{MAINTENANCE}} y \texttt{\textbf{COOLER}} se escribe como \texttt{\textbf{||COOLING\_TOWER =
(MAINTENANCE||COOLER).}}

Además, FSP posee palabras reservadas que se colocan antes de la definición de un proceso que fuerzan a la herramienta
MTSA a realizar una operación mas compleja al proceso. Un caso de estos, es la palabra reservada
\texttt{\textbf{minimal}}, la cual, hace que MTSA construya un LTS minimal que respeta la semántica equivalente o la
palabra reservada \texttt{\textbf{deterministic}}, que construye un LTS minimal con respecto a las trazas.

FSP también permite definir propiedades FLTL. Un fluent que marca aquellos estados donde la bomba esta apagada puede ser
expresada en lenguaje FSP mediante el siguiente código: \texttt{\textbf{fluent IsStopped = <stopPump,startPump>\ initially 0}}.
Como dijimos anteriormente, la bomba empieza encendida, por lo tanto IsStopped es inicialmente falso, pasa a ser
verdadero cuando sucede la acción \texttt{\textbf{stopPump}} y falso nuevamente cuando la acción \texttt{\textbf{startPump}} sucede.

Finalizando, FSP nos otorga facilidad para especificar LTSs y FLTL formulas. Este lenguaje es el que utilizaremos en los
siguiente capítulos para definir modelos que representan ambientes y objetivos. 
