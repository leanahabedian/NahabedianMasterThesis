\section{Lógica Lineal Temporal de Flujos (Fluent Linear Temporal Logic)}

La Lógica Lineal Temporal (LTL) esta siendo ampliamente usada en la ingeniería de los requerimientos
\cite{1347546,Giannakopoulou:2003:FMC:940071.940106,879820,Letier:2002:ATG:581339.581353}. La motivación para escoger a
las LTL de flujos es que estas proveen un framework uniforme para especificar propiedades basados en estados sobre
modelos basados en eventos \cite{Giannakopoulou:2003:FMC:940071.940106}. Fluent Linear Temporal Logic (FLTL) \cite{Giannakopoulou:2003:FMC:940071.940106}
es una lógica de tiempo lineal, temporal, para razonar acerca de flujos. Un \emph{flujo} $Fl$ es definido por un par de
conjuntos y un valor booleano: $Fl = <I_{Fl},T_{Fl},Init_{Fl}>$, donde $I_{Fl} \subseteq Act$ es el conjunto de acciones
iniciadoras, $T_{Fl} \subseteq Act$ es el conjunto de acciones finalizadoras y $I_{Fl} \cap T_{Fl} = \emptyset$. Un
flujo puede ser inicializado con valor true o false indicado por $Init_{Fl}$. Toda acción $\ell \in Act$ induce un flujo,
que notaremos $\ell = <\ell, Act \ \{\ell\}, false>$. Por último, el alfabeto de un flujo es el que se obtiene mediante la
unión del conjunto de acciones iniciadoras y el conjunto de acciones finalizadoras.

Sea $\mathcal{F}$ el conjunto de todas las posibles flujos sobre $Act$. Una formula FLTL esta se define inductivamente
utilizando los conectores booleanos estandar y operadores temporales como el $\mathbf{X}$ (próximo), $\mathbf{U}$ (antes
fuerte) de la
siguiente manera:

\begin{center}
\begin{equation}
    \varphi ::= Fl\ |\ \neg\varphi\ |\ \varphi\ \lor\ \psi\ |\ \mathbf{X} \varphi\ |\ \varphi\ \mathbf{U}\ \psi
\end{equation}
\end{center}

\noindent donde $Fl \in \mathcal{F}$. Para comodidad sintáctica, vamos a introducir las operaciones de $\land$,
$\Diamond$ (eventualmente) y $\Box$ (siempre). Sea $\Pi$ el conjunto de trazas infinitas sobre $Act$, diremos que la
traza $\pi = \ell_0,\ell_1,...$ satisface un flujo $Fl$ en la posición $i$, notado $\pi,i\vDash Fl$, si y solo si una de las
siguientes condiciones es válida:

\begin{itemize}
    \item $Init_{Fl} \land (\forall j \in \mathbb{N}: 0 \leq j \leq i \rightarrow \ell_k \notin T_{Fl} )$
    \item $\exists j \in \mathbb{N}: (j \leq i \land \ell_j \in I_{Fl}) \land (\forall k \in \mathbb{N}: j < k \leq i
    \rightarrow \ell_k \notin T_{Fl})$
\end{itemize}

\noindent Dada una traza infinita $\pi$, la fórmula que satisface $\varphi$ en la posición $i$, denotada como $\pi,i
\vDash \varphi$, es definida a continuación como se muestra en la semántica para el operador de satisfacción:

\begin{center}
\begin{tabular}{p{2cm}p{0.5cm}l}
    $\pi,i \vDash Fl                        & \triangleq    & \pi,i \vDash Fl$\\
    $\pi,i \vDash \neg\varphi               & \triangleq    & \neg(\pi,i \vDash \varphi)$\\
    $\pi,i \vDash \varphi \lor \psi         & \triangleq    & (\pi,i\vDash\varphi)\lor(\pi,i\vDash\psi)$\\
    $\pi,i \vDash \mathbf{X}\varphi         & \triangleq    & \pi,1 \vDash \varphi$\\
    $\pi,i \vDash \varphi\mathbf{U}\varphi  & \triangleq    & \exists j \geq i: \pi,j \vDash \psi\land\forall i \leq k < j:
    \pi,k \vDash \varphi$
\end{tabular}
\end{center}

Diremos que $\varphi$ se cumple en $\pi$, denotado como $\pi \vDash \varphi$, si $\pi,0 \vDash \varphi$. Una fórmula
$\varphi \in$ FLTL es cierta si un LTS $E$ (denotado como $E \vDash \varphi$) si éste es cierto en toda traza infinita
producida por $E$. 
