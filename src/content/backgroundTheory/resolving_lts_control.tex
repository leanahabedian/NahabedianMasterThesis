\section{Resolviendo el problema de control LTS SGR(1)}

En esta sección explicaremos como una solución para un problema de control SGR(1) puede ser obtenida por construcción
utilizando técnicas existentes de síntesis de controladores (basados en estados), llamados GR(1). \cite{Piterman}

La construcción de la máquina para un problema de control LTS SGR(1) esta divido en dos pasos. Primero, se crea un juego GR(1)
$G$ en representación del ambiente $E$, las asunciones $A_s$, los objetivos $O$ y el conjunto de acciones controlables
$A_C$. Como segundo paso, se elabora una solución $(\sigma,u)$ al juego GR(1) para construir una máquina $M$ (i.e un
controlador LTS) para $\mathcal{E}$. Esta solución al problema de control LTS SGR(1) $\mathcal{E}$ existe, si y solo si,
existe una solución al juego GR(1) $G$. Luego, podremos afirmar que el controlador LTS $M$ creada a partir de
$(\sigma,u)$ es una solución a $\mathcal{E}$.

\subsection{Control LTS SGR(1) a juegos GR(1)}

texto

\subsection{Traduciendo la estrategia a un Controlador LTS}

texto

\subsection{Algoritmo}

algoritmo
