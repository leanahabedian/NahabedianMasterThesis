\section{Modelando objetivos para controladores actualizables}

Agregamos un conjunto de palabras reservadas a FSP para poder soportar objetivos actualizables. En la figura
\ref{TEMPLATE1} mostramos el código FSP necesario para síntesis de controladores actualizables para el ejemplo del reactor nuclear
presentado en la sección \ref{}.

El operador \texttt{\textbf{updatingController}} devuelve, si existe, un controlador que satisface una especificación
acerca de los nuevos requerimientos. Necesitamos en esta declaración, además de los nuevos requerimientos, suministrar
la información necesaria acerca de cual es el controlador actual, el modelo del ambiente actual, el modelo del ambiente
nuevo y un conjunto de pares de flujos donde cada elemento indicará correspondencia de estados del ambiente actual al
ambiente nuevo como dijimos en la sección \ref{}. Por ejemplo, en la figura \ref{TEMPLATE1} podemos observar que
al usar la palabra reservada \texttt{\textbf{updatingController}} configuramos BLA1 como el controlador actual, BLA2
como modelo del ambiente actual, BLA3 como modelo del ambiente nuevo, \texttt{\textbf{UpdateSpec}} como la nueva
especificación y \texttt{\textbf{\{\{RequestPending,RequestPending\}, \{IsStopped,IsStopped\}\}}} es el conjunto de
flujos.

Los objetivos nuevos están definidos mediante el operador \texttt{\textbf{controllerSpec}}. La palabra reservada
\texttt{\textbf{safety}} permite definir requerimientos de seguridad (\emph{safety}). Aquí incluiremos las formulas
LTL descritas en la sección \ref{}

La sección de \emph{liveness} esta definida ... 
