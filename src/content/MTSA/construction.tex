\section{Construcción}

En MTSA, los modelos son definidos mediante una extensión del lenguaje de Procesos de Estados Finitos (FSP). Dicho
lenguaje es un lenguaje textual centrado en la construcción composicional de modelos complejos que originalmente fue
usado para describir LTSs.

Los FSP incluyen varios operadores tradicionales para describir modelos de comportamiento, de los cuales destacaremos el
prefijo de acción ($\rightarrow$), elección ($|$), composición secuencial (;), composición paralela ($\|$) y mezcla.
La semántica de la mezcla es tal que dada dos descripciones parciales del mismo componente, el operador de mezcla
devuelve una LTS que combina la información provista por las descripciones parciales originales.

Es necesario destacar que construir los modelos que son compuestos sigue siendo una tarea dificultosa que requiere de un
intenso trabajo y un grado considerable de experiencia. Para mitigar este problema, MTSA también provee la funcionalidad
que permite sintetizar modelos de comportamiento de forma automática, a partir de especificaciones declarativas de los
requerimientos, escenarios y casos de uso.

La palabra clave \emph{constraint} de MTSA se usa de manera conjunta con las propiedades de seguridad (\emph{safety})
formalizadas haciendo uso de Lógica Lineal Temporal de Flujos (FLTL). Para una declaración de tipo \emph{constraint},
MTSA construye automáticamente el modelo de LTS que caracteriza a todos los modelos LTS libres de \emph{deadlock} que
satisfacen la fórmula FLTL. Al sintetizar y mezclar modelos LTS obtenidos con definiciones FLTL, se puede construir de
forma iterativa una LTS que caracteriza a la cota superior de los sistemas de comportamiento esperados.

La palabra clave \emph{abstract} de MTSA puede aplicarse a procesos FSP. Su semántica es tal que el modelo resultante es
la LTS de menor refinamiento que garantiza el comportamiento requerido por los procesos FSP. Esta palabra clave,
utilizada en conjunción con los procesos FSP que modelan el comportamiento descripto en la especificación del escenario,
provee una LTS que caracteriza a todas las implementaciones que satisfacen dicha especificación.
