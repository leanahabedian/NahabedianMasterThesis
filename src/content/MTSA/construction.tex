\section{Construcción}

En MTSA, los modelos son definidos mediante una extensión del lenguaje de Procesos de estados finitos (FSP). Dicho
lenguaje es un lenguaje textual centrado en la construcción composicional de modelos complejos que originalmente fue
usado para describir LTSs.

FSP incluye varios operadores tradicionales para describir modelos de comportamiento, de los cuales destacaremos el
prefijo de acción ($\rightarrow$), elección ($|$), composición secuencial (;), composición paralela ($||$) y mezcla.
La semántica de la mezcla es tal que dada dos descripciones parciales del mismo componente, el operador de mezcla
devuelve una LTS que combina la información provista por las descripciones parciales originales.

Es necesario destacar que construir los modelos que son compuestos sigue siendo una tarea dificultosa que requiere de un
intenso trabajo y un grado considerable de experiencia. Para mitigar este problema, MTSA también provee la funcionalidad
que permite sintetizar modelos de comportamiento de forma automática a partir de especificaciones declarativas de los
requerimientos, escenarios y casos de uso.

 POR AHORA HASTA ACA ESTA BIEN
