\section{Discusión y trabajo futuro}

El problema de actualización dinámica ha sido estudiado extensivamente y hay una gran cantidad de problemas distintos
que deben ser abordados en función del dominio de aplicación, tecnología y la forma en que se efectúa la actualización
(vea \cite{SMR:SMR1556} para más información). El grueso del esfuerzo en actualización dinámica asume que no hay cambios
de especificación y por lo tanto el mismo comportamiento es esperado (por ejemplo en \cite{mx:icse13}), o que la
especificación es genérica (como en \cite{Shen:2005:TUF:1095430.1081720}, \cite{5551162}, \cite{1167829},
\cite{4221625}, \cite{485222}, \cite{60317}) y no proporcionada por el usuario. Ejemplos de este último, aparte de
asegurar que la actualización no desemboca un fallo, asegura type safety (ej. \cite{Subramanian08dynamicsoftware}) y
data isolation entre versiones \cite{Stoyle07mutatismutandis:}. Quiescence \cite{60317} y nociones relacionadas (ej.
\cite{4359466},  \cite{Anderson:2009:MPM:1656437.1656448}, \cite{485222}) no requieren una representación explicita de
las propiedades a ser preservadas, pero fue utilizado en conjunto con técnicas que garantizan consistencia genérica de
semántica (ej. \cite{5546542}). 

La necesidad de una actualización por medio de la definición de propiedades por parte del usuario han sido reconocidas
en \cite{Baresi:2010:DBD:1882362.1882367}. En \cite{Hayden:2012:SVC:2189314.2189336}, se considera la especificación de
propiedades de actualización, pero posee un enfoque en verificar si el programa satisface estas propiedades, en vez de
automáticamente sintetizar la actualización para satisfacer estas propiedades. 

Síntesis, estrategia operacional para construir un autómata que satisface una especificación dada, ha sido usado
extensivamente para garantizar código que es correcto por construcción (por ejemplo en
\cite{Greenyer:2013:ISC:2491411.2491445}). La naturalidad automática de la síntesis logra poder aplicar esta técnica no
sólo en el momento en el que el diseño se lleva a cabo sino que también en tiempo de ejecución, generando un sistema
adaptable. Dicha evolución no esta limitada exclusivamente para sistemas adaptables como describimos en la etapa de
introducción y motivación (ver sección \ref{motivation}). Por ejemplo, en \cite{Pelliccione20082237} el problema de
evolucionar componentes conjuntos es resuelto sintetizando ``glue code'' (por ejemplo, controladores). Si bien la
síntesis se ejecuta sin frenar el sistema, el nuevo controlador sólo puede reemplazar al viejo cuando el sistema entra
en ``quiescence''.

La síntesis requiere algún tipo de especificación desde la que, a través de diferentes técnicas de razonamiento, producir
una solución. El resultado de sintetizar es correcto sólo en la medida en que la especificación es válida. Por lo tanto,
técnicas de síntesis no son, en principio, resistentes a errores en la especificación o a ambientes que evolucionan y
divergen de la especificación. El trabajo descrito en esta tesis es también susceptible a especificaciones invalidas. En
el dominio de sistemas adaptables se han estudiado distintos enfoques que pueden detectar y resolver ciertas situaciones
(por ejemplo en \cite{DBKMSU14} y \cite{Vromant:2011:ICL:1988008.1988037}) e incluso, aprender nuevas especificaciones
en tiempo de ejecución. El enfoque descrito en este documento puede ser combinado con dichas técnicas.

Una suposición de este trabajo es que es posible controlar cambios del ambiente. En otras palabras, si la nueva
especificación introduce diferentes características en el ambiente, como serían nuevos componentes, cambios en el protocolo
de llamada de un componente, o deshabilitar un componente, es el controlador quien puede decidir cuando estos cambios
ocurren. El controlador controla una infraestructura intermedia que puede cargar, descargar o cambiar componentes en
tiempo de ejecución. Esto permite al controlador a planificar un cambio y darle más libertad para encontrar una
estrategia que puede satisfacer el cambio requerido. Por otro lado, en muchas situaciones, puede suceder un cambio no
anunciado en el ambiente y actualizar el controlador para acomodar este cambio es deseado.

Por ejemplo, si la conexión entre el UAV y la base puede perderse, y al suceder esto, la interfaz que el controlador de UAV
maneja queda deshabilitada. En este caso, la actualización de este controlador debe producirse inmediatamente y puede
ser imposible continuar garantizando los objetivos actuales o los nuevos. En \cite{DBKMSU14} presentan un enfoque donde las
garantías funcionales del controlador se degradan elegantemente para estos casos. Por otro lado, la técnica requiere que
el controlador y la especificación del nivel de degradación preserve una relación de refinación entre el controlador
actual y la especificación. Dicho requerimiento puede ser restrictivo.

El trabajo realizado en esta tesis, puede ser visto como una generalización de una degradación progresiva presentada en
\cite{DBKMSU14}. Decimos esto, ya que la no controlabilidad del cambio de ambiente, puede en nuestro enfoque manejarse teniendo
que la especificación de transición $T$ requiera que el evento $startNewSpec$ ocurra inmediatamente después de
$beginUpdate$, sin embargo, el ambiente viejo $E$ no es requerido para obtener refinamientos de $E$ como en
\cite{DBKMSU14}

En este trabajo, limitamos la expresividad de nuestro enfoque a objetivos de \emph{safety} (ver Definición
\ref{update_goals_def}). Esto hace a esta presentación simple e incluso permite una resolución de complejidad lineal al
problema de control de actualización de controladores dinámicamente siempre y cuando el ambiente es determinístico. Sin
embargo, podríamos permitir mayor expresividad en $G$, $G'$ y $T$ sin incurrir del todo en la penalidad de resolver un
problema de control (2EXPTIME-COMPLETE). Es posible, reformular la definición \ref{update_goals_def} para permitir que
las especificaciones $G$, $G'$ y $T$ puedan ser de la forma $\Box \Diamond G$, $\Box \Diamond G'$ y $\Box \Diamond T$.
Este criterio de aceptación de Büchi extiende la expresividad manteniendo la complejidad en orden polinomial (esto lo
conseguimos redefiniendo $G_u$ a $\Box \Diamond [(G \lor OldSpecDropped) \wedge (G' \lor \neg NewSpecEnsured) \wedge
(\neg BeginUpdate \lor NewSpecEnsured) \wedge (\neg BeginUpdate \lor  OldSpecDropped) \wedge  T]$. Donde
$OldSpecDropped$ es el fluent definido como $\langle \{stopOldSpec\},\emptyset,\bot\rangle$, $NewSpecEnsured$ es el fluent
definido como $\langle \{startNewSpec\},\emptyset,\bot\rangle$ y $BeginUpdate$ es el fluent definido como $\langle
\{beginUpdate\},\emptyset,\bot\rangle$). Se necesitan futuras investigaciones para analizar si, por ejemplo, las
especificaciones SGR(1) \cite{D'ippolito:2013:SNE:2430536.2430543} en el problema de control de actualización de
controladores dinámicos, sólo puede ser resuelto en tiempo polinómico.

La complejidad lineal del problema de control de actualización de controladores dinámicamente cuando se restringe a
propiedades de \emph{safety} provee un argumento analítico para la escalabilidad. Sin embargo, queda por hacer una
validación experimental, y en particular evaluar la necesidad práctica de introducir el no-determinismo, lo que puede
producir que la complejidad se vaya a exponencial en peor caso. Los problemas de control en los casos de estudio
descritos en la sección \ref{cases_of_study} fueron resueltos usando MTSA que provee algoritmos más complejos que los
necesitados (polinomial para SGR(1) en vez de linear para formulas de obligación). Además, la estrategia para resolver
el problema de actualización definida en esta tesis, que construye un controlador que controla las tres fases de
actualización (ver Figura \ref{update_environment}), prioriza el entendimiento de dicha técnica. Para reducir la
complejidad (particularmente para el caso no lineal de fórmulas de Büchi), es posible producir un controlador de
actualización composicionalmente, computando primero estados en el nuevo ambiente desde los cuales los nuevos objetivos
pueden ser garantizados, y luego, computando un controlador que puede satisfacer alcanzar uno de esos estados luego de
que $beginUpdate$ ocurra, mientras se satisfacen los requerimientos de transición $T$.

Tenga en cuanta, que no discutimos hasta ahora que pasa si otra actualización dinámica es deseada luego de que se
efectúe la primera. Una visión \emph{naive} para concretar ésto sería considerar a $C_u$, computado en la primera
actualización, como controlador para la segunda. Esta idea sería ineficiente ya que  $C_u$ contiene, como lo muestra la
Figura \ref{update_controller}, el comportamiento entero del controlador original $C$ más el control de la fase de transición y el
control para la nueva especificación. Usar a $C_u$ como $C$ para la segunda actualización dinámica y aplicar la misma
técnica tantas veces sea necesario, llevaría a obtener un controlador que posee el historial de todos los controladores
y las actualizaciones realizadas en el sistema. Ésto contrae problemas de escalabilidad. Esta situación puede ser
evitada ya que una vez que la actualización se ha ``estabilizado'' con la nueva especificación (esto es, el controlador
de actualización efectuó la acción $startNewSpec$ y $stopOldSpec$), el controlador $C_u$ garantiza no visitar estados
que correspondan a la fase inicial (ésto puede verse en el super-estado $C_{G'}$ de la Figura \ref{update_controller}).
Por lo tanto, para la nueva actualización, en vez de considerar por completo todos los estados de $C_u$, es posible
considerar la componente conexa de $C_u$ que incluye al estado actual de $C_u$ (lo que es a lo sumo, todos los estados
que pertenecen al super-estado de $C_{G'}$ de la Figura \ref{update_controller}).

En la sección \ref{cases_of_study} comparamos nuestro enfoque con \cite{6224401} y
\cite{PanzicaLaManna:2013:FCC:2487336.2487349} mediante casos de estudio. Como discutimos, una ventaja del trabajo de
\cite{6224401} es que no le fue necesario especificar requerimientos de transición, requiriendo que la actualización sea
equivalente a una actualización offline. La desventaja, es que si el sistema no puede retornar a un estado inicial y no
exhibe comportamiento compatible con la nueva especificación desde el último estado inicial, entonces la actualización
no podrá ser efectuada. El caso de estudio del UAV de la sección \ref{buscador_UAV} ejemplificamos ésto. También
generalizamos la relación entre historias entre estados de la especificación vieja y de la nueva usadas en
\cite{6224401} y \cite{PanzicaLaManna:2013:FCC:2487336.2487349}, permitiendo al usuario especificar un mapeo de estados.

Finalizando, como discutimos en el capítulo \ref{validation} el criterio de actualización correcta definido en
\cite{6224401} puede ser expresado como un requerimiento de transición, pero hay una pequeña diferencia en los
resultados obtenidos. Nuestro controlador de actualización forzará al sistema a que llegue al estado inicial,
\emph{garantizando} que la actualización va a ocurrir. Mientras que en \cite{6224401} hay una \emph{asunción} de que el
controlador actual y su ambiente van a ser cooperar a que el estado inicial eventualmente sea visitado, es decir, no
garantiza que la actualización ocurra.
