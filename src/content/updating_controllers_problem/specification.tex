\section{Especificación}

En capítulos anteriores mostramos ejemplos en los cuales nos referíamos a la actual y nueva especificación como
entidades monolíticas (las llamaremos $S$ y $S'$ respectivamente). Sin embargo, para presentar la actualización dinámica
de controladores como un problema de control supondremos que la especificación $S$ está dividida en tres partes. La
primera, es la descripción de la interfaz del controlador dada como un conjunto de etiquetas $A$ que representa las
acciones controlables del controlador. La segunda, es una descripción operacional $E$, en forma de LTS, que describe el
comportamiento del ambiente en cuanto a acciones controlables y monitoriables. La tercera, son los objetivos del
controlador $G$, expresados como una formula FLTL.

Tenga en cuenta que como el estado inicial de $E'$ depende el estado actual de $E$, el ingeniero debe proveer una relación de estados
$M$ de $E$ a $E'$. Omitiremos referirnos a $M$ hasta que lleguemos a la sección donde formalizaremos todos estos
conceptos para mantener la presentación simple.

En consecuencia, el problema general que apuntamos resolver puede ser planteado de la siguiente manera:
Un sistema adaptable mediante un controlador $C$ que controla un conjunto de acciones $A$ y a su vez satisface los
objetivos $G$ para un ambiente $E$. El usuario desea cambiar dinámicamente el controlador que se está ejecutando para
empezar a satisfacer el nuevo objetivo $G'$ en el nuevo ambiente $E'$ controlando un conjunto de acciones $A'$ y a su
vez satisfacer los requerimientos de transición $T$.

Por ejemplo, en el caso de estudio del buscador UAV de vida salvaje de la sección \ref{buscador_UAV}, el conjunto de acciones
controlables para la especificación original ($A$) incluye $start$, $return2base$, $scan$, $picture$ y $goto$. Las
acciones controlables para la nueva especificación ($A'$) es un conjunto más extenso ya que provee nuevas capacidades para el
controlador como por ejemplo $follow$.

Los modelos de los ambientes $E$ y $E'$ pueden ser construidos mediante la composición paralela de varios modelos LTSs
que describen diferentes aspectos del comportamiento del ambiente. Por ejemplo podríamos definir un LTS que describa el
comportamiento de un tren que va atravesando distintas secciones de una vía. Por otro lado, tenemos otra LTS que
describe la comunicación que el tren realiza con la barrera electrónica para identificar si está habilitado para cruzar.
Para definir por completo el ambiente, necesitamos componer en paralelo ambos LTS. 

