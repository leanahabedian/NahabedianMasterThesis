\section{Especificación}

En capítulos anteriores mostramos ejemplos en los cuales nos referíamos a la actual y nueva especificación como
entidades monolíticas (las llamaremos $S$ y $S'$ respectivamente). Sin embargo, para presentar la actualización dinámica
de controladores como un problema de control asumiremos que la especificación $S$ esta dividida en tres partes. La
primera es la descripción de la interfaz del controlador dada como un conjunto de etiquetas $A$ que representa las
acciones controlables del controlador. La segunda es una descripción operacional $E$, en forma de LTS, que describe el
comportamiento del ambiente en cuanto a acciones controlables y monitoriables. La tercera son los objetivos del
controlador $G$, expresado como una formula FLTL.

Note que como el estado inicial de $E'$ depende el estado actual de $E$, el ingeniero debe proveer un mapeo de estados
$M$ de $E$ a $E'$. Omitiremos referirnos a $M$ hasta que lleguemos a la sección donde formalizaremos todos estos
conceptos para mantener la presentación simple

En consecuencia, el problema general que apuntamos a resolver puede ser planteado de la siguiente manera: La
comportamiento de un sistema adaptable están siendo ejecutadas
