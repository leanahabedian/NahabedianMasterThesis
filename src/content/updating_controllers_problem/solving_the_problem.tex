\section{Resolviendo el problema de actualización dinámica de controladores}

Resolver un problema de control LTS (como el definido en la definición \ref{LTS_control} en el capitulo
\ref{background}) por cada propiedad FLTL es 2EXPTIME complete \cite{Pnueli:1989:SRM:75277.75293}. El problema de
actualización dinámica de controladores es una instancia especifica del problema de control LTS general. En efecto, dada
la estructura de $G_u$, su resolución puede ser computada bajo limites de menor complejidad.

Bajo la asunción de que $G$, $G'$ y $T$ son propiedades de seguridad, $G_u$ puede ser codificado como propiedades de
obligación (por ejemplo disyunciones de afirmaciones de seguridad y alcanzabilidad, $\bigwedge^n_{i=1}(\Box I_i \lor
\Diamond R_i)$). Los problemas de control LTS con objetivos de obligación pueden ser resueltos en tiempo lineal para
modelos de ambientes determinísticos. Para ambientes no determinísticos, un especializado subconjunto de construcciones
pueden usarse para producir versiones determinísticas, sin embargo, podría crecer exponencialmente dependiendo del grado
de no determinismo que exista.

Es simple de ver que el objetivo del problema de control de actualización dinámica de controladores, $G_u$, puede ser
reformulado como afirmaciones de obligación. Las formulas 1) a 3) en definición \ref{update_goals_def} son propiedades
de seguridad (tenga en cuenta que $W$ es un ``weak until'' y por lo tanto es una propiedad de seguridad). Una propiedad
de seguridad $I_j$ puede simplemente ser codificada como $\Box I_j \lor \Diamond \bot$. Las otras dos formulas de la
definición \ref{update_goals_def} son de la forma $\Box(p \Longrightarrow \Diamond q)$ y pueden ser codificadas como
$\Box \neg UpdateBegan \lor \Diamond (UpdateBegan \wedge NewSpecStarted)$ y $\Box \neg UpdateBegan \lor \Diamond
(UpdateBegan \wedge OldSpecStopped)$ donde los flujos $NewSpecStarted$, $OldSpecStopped$ y $UpdateBegan$ modelan que sus
respectivos eventos han sucedido alguna vez en el pasado. Formalmente estarían definidos como $\langle{startNewSpec},
\emptyset, \bot \rangle$, $\langle{stopOldSpec}, \emptyset, \bot \rangle$ y $\langle{beginUpdate}, \emptyset, \bot
\rangle$.

El problema de control de la actualización dinámica de controladores puede tener o no una solución. La existencia de la
esta solución significa que el controlador resultante $C_u$ garantiza que: se comporte inicialmente como el controlador
actual $C$ bajo la especificación del ambiente actual $E$ y acepta en cualquier punto el comando $beginUpdate$; asegura
una transición correcta, satisfaciendo $T$, preservando el objetivo viejo $G$ y conduciendo al sistema a un punto en el
cual ($startNewSpec$) el nuevo ambiente $E'$ puede ser reemplazado y el nuevo objetivo $G'$ puede ser garantizado.

La no existencia de una solución a este problema de control puede significar dos posibles escenarios. El primero es que
el nuevo objetivo bajo las asunciones del nuevo ambiente no puede ser alcanzados por el controlador (esto es, el
problema de control definido por $E'$ y $\Box G'$ no tiene solución para ningún estado inicial de $E'$). Esta es una
situación extrema donde el controlador tiene poco por hacer con la actualización que se le pide. El segundo, asumiendo
que el problema de control definido por $E'$ y $\Box G'$ tiene solución para algunos estados de $E'$, no es solo que la
transición de una especificación a la otra no pueda satisfacer la propiedad $T$, sino que también puede suceder que sea
imposible alcanzar un estado de $E'$ desde el cual $G'$ valga. Este segundo escenario puede producirse, por ejemplo,
porque el requerimiento de transición $T$ es excesivamente restrictiva o porque el controlador tiene capacidades
insuficientes para asegurar alcanzar un estado desde el cual el nuevo ambiente puede garantizar la validez del nuevo
objetivo. Mas generalmente, tendremos el problema cuando tenemos una combinación de especificaciones muy estrictas ($E$,
$G$, $E'$, $G'$ y $T$) y falta de control del controlador sobre los eventos del ambiente. En el capítulo \ref{cases_of_study}
discutiremos las causas de no poder resolver el problema de actualización dinámica de controladores para algunos
ejemplos específicos.
