\section{La actualización dinámica de controladores como un problema de síntesis de controladores}

Una primera aproximación intuitiva para implementar la actualización de controladores dinámicamente puede ser
construyendo dos controladores adicionales al controlador actual $C$. El primero es un controlador $C'$ que puede
satisfacer los objetivos $G'$ para el nuevo ambiente $E'$ controlando acciones $A'$. El segundo es un ``controlador de
transición'' $C^*$ que controla el traspaso del controlador actual $C$ al nuevo controlador $C'$ satisfaciendo los
requerimientos de transición $T$.

Si bien es conceptualmente elegante, el enfoque de los tres controladores es ingenuo ya que estos controladores están
intrínsecamente relacionados. La nueva especificación solo puede ser alcanzada por un controlador $C'$ desde estados
específicos ($I_{C'}$). Estos estados iniciales de $C'$ necesitan ser calculados y considerados como estados finales del
``controlador de transición'' $C^*$. Luego, los estados desde donde el ``controlador de transición'' puede alcanzar
($I_{C'}$) necesitan ser computados ($I_{C^*}$). Finalmente, nos queda analizar si $C$ puede ser extendido para
garantizar que alcance algún estado en $I_{C^*}$ sin violar sus objetivos ($G$).

La interacción entre estos tres controladores y la necesidad de generar una técnica para computar $C'$ y $C^*$ puede ser
obtenida mediante la resolución de un problema de control que produce un solo controlador (el cual vamos a llamar $C_u$)
que ejecuta acciones simulando las tres faces, primero simulando a $C$, luego a $C^*$ y finalmente a $C'$. Ahora
pasaremos a explicar como la actualización de controladores dinámica puede ser expresada como un problema de control que
abarca estas tres fases.

El problema de control para la actualización de controladores dinámicos, como cualquier otro problema de control,
necesita de un modelo del ambiente, el cual llamaremos $E_u$, un objetivo, el cual llamaremos $G_u$, y un conjunto de
acciones, $A_u$. El objetivo $iG_u$, lo definiremos en términos de $G$, $G'$ y $T$ mas los eventos $stopOldSpec$,
$startNewSpec$ y $beginUpdate$. El ambiente $E_u$ estará definido en base a $E$, $E'$ y $C$. El conjunto de acciones
$A_u$ estará definido en base a $A$ y $A'$.

