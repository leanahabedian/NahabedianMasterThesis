\section{Casos de Estudio}
\label{cases_of_study}

\subsection*{Configuración experimental}

Todos los casos de estudios se ejecutan usando una extensión de la herramienta MTSA \cite{4639371} que nos da soporte
para especificar LTSs y propiedades, usando una notación textual, álgebra de procesos y FLTL. La herramienta también nos 
da soporte para síntesis de controladores para problemas de control SGR(1), los cuales son estrictamente más caros de lo
que es requerido para problemas de control en la actualización de controladores dinámicamente. La herramienta fue 
extendida para automáticamente computar $A_u$, $G_u$ y $E_u$, y resuelve el problema de control de la actualización
dinámica de controladores. La versión extendida de la herramienta y la especificación completa de todos los casos de
estudio pueden verse en \url{http://sourceforge.net/projects/mtsa/}.

\subsection{Planta de energía nuclear}
\label{power_plant}

En \cite{PanzicaLaManna:2013:FCC:2487336.2487349} se analiza un controlador para un sistema de refrigeración de una
planta de energía nuclear. El controlador actual debe ejecutar servicios de mantenimiento primero apagando la
bomba que refrigera la planta y luego reiniciarla. El nuevo controlador no necesita parar la bomba y reiniciarla para
efectuar el mantenimiento. Hay también un invariante del sistema que indica que la bomba de refrigeración no puede estar
apagada indefinidamente (esto puede llevar a un accidente devastador). 

Los autores muestran que si una actualización se lleva a cabo en un estado en donde el controlador actual tiene la bomba
apagada, entonces el nuevo controlador puede no reiniciar la bomba, generando un accidente. Esto nos muestra dos
problemas. Primero, que la forma segura de preservar el invariante del sistema es requerir que la actualización preserve
el comportamiento como si el controlador actual haya sido reemplazado desde su estado inicial (cuando la bomba es
reiniciada). Esto es lo que los autores aseguran en \cite{6224401}. Segundo, que las restricciones de cuándo una
actualización de \cite{6224401} es suavizada como en \cite{PanzicaLaManna:2013:FCC:2487336.2487349}  puede generar
fallas y consecutivamente ellos recomiendan que ``el diseño de controladores actualizables dinámicamente basados en el
criterio débil de  actualización incluye una subsiguiente validación de los controladores''.

%\subsection{RailCab}

%El sistema de RailCab es un sistema mecatrónico que fue desarrollado en la Universidad de Paderborn \cite{railcab} donde
%los vehículos autónomos deben coordinar de que manera transportar pasajeros o bienes cada vez que le es solicitado. El
%sistema que se discute en \cite{6224401} como caso de estudio base es controlar RailCabs cuando está acercándose a un
%cruce de vías. El RailCab puede monitorear eventos como $endOfSection$ que determinan que esta entrando a un cruce de
%vía, y otros eventos que indican si esta habilitado para usar el freno o si tiene que utilizar el freno de emergencia,
%que los llamaremos $lastBrake$ y $lastEmergencyBrake$. El RailCab controla el freno ($brake$) y el freno de emergencia
%($emergencyBrake$) e incluso puede recibir respuestas a pedidos que controla como $requestPermissionToEnter$.

%La especificación del controlador actual requiere que el RailCab entre al cruce de vías sólo si se le ha garantizado el
%permiso para hacerlo. La especificación también describe varias asunciones de que la respuesta al pedido de permiso es
%contestada antes de que el RailCab llegue a la sección de $lastEmergencyBrake$, siempre y cuando, el pedido haya sido
%realizado antes de llegar a la zona de $lastBrake$. Como ejemplo de una nueva especificación, en \cite{6224401} discuten
%hacer un objetivo más fuerte requiriendo que el RailCab envíe otro pedido preguntando si las barreras están funcionando
%antes de entrar al cruce de vía. También refuerzan la hipótesis del ambiente en relación a cuando la respuesta a
%$checkCrossingStatus$ puede ser recibida.

\subsection{Buscador UAV de vida salvaje}
\label{buscador_UAV}

Este caso de estudio consta de un robot UAV (unmanned aerial vehicle), un típico caso de sistema adaptable, que posee un
controlador para buscar vida salvaje en una área de preservación. El UAV cuenta con un algoritmo de reconocimiento de
vida salvaje que requiere que el UAV vuele en el rango de 40 a 50 metros y trasmita una foto por cada posible objetivo y
debe volver a la base cuando la señal de batería baja se dispara. Durante la misión, se decide que el UAV debe ahora
inmediatamente seguir al primer posible objetivo que encuentre en vez de continuar su búsqueda. Además, deberá trasmitir
una foto de este animal cada 5 minutos. La nueva misión, requiere que el UAV vuele en un rango entre 20 y 30 metros de
altura y use un algoritmo de reconocimiento mejorado. Los módulos del software nuevo van a estar cargados en el UAV y el
controlador debe ser actualizado para lograr sus nuevos objetivos usando los nuevos módulos.

Es importante destacar que los rangos de alturas no solapados implican que un período de transición serán requeridos
entre las dos misiones donde el UAV deberá volar entre 30 a 40 metros de altura y ninguna especificación se satisface.
Para una representación gráfica en una linea de tiempo de este caso de estudio vea la figura \ref{transition}

A su vez, existen otros requerimientos que pueden ser relevantes para el período de transición. Por ejemplo, la nueva
especificación no tiene en cuenta lo que sucederá con las imágenes que están siendo transmitidas por cada animal
encontrado. Consecuentemente, una actualización que sucede entre que un animal es encontrado y la transmisión de la foto
puede llevar a perder la foto. Por lo tanto, un requerimiento para el período de transición puede incluir que a la hora
de dejar de satisfacer la especificación vieja, no haya imágenes pendientes para transmitir.


\subsection{Production Cell}

Consideramos varios escenarios de actualización para el caso de estudio de Production Cell presentado en
\cite{Lewerentz:1995:646391}. El caso de estudio es sobre una fábrica que manofactura diferentes tipos de productos,
donde cada uno de ellos, requiere un proceso de producción particular. El sistema de producción de la fábrica debería
adaptar su proceso de producción al número de factores como la cantidad de herramientas disponible y la especificación
de como procesar cada tipo de producto. 

Además de las herramientas, la fábrica tiene una bandeja de entrada, una bandeja de salida y un brazo robótico. El brazo
robótico es el encargado de mover los productos entre las herramientas y las bandejas. Los productos en crudo llegan a
la bandeja de entrada, para que el brazo robótico controle cada producto respetando la especificación y colocando el
producto final en la bandeja de salida.

\subsection{Cuadcóptero}

El propósito de este caso de estudio fue experimentar la técnica de actualización dinámica más allá de la especificación
y tareas de síntesis discutidos en los casos de estudio previos. El enfoque, entonces, de este caso es poder analizar un
controlador resultante ejecutando dicha solución en una infraestructura de adaptación (enactment), tomando un dominio de
aplicación especifico. Usamos una infraestructura reportada en \cite{Braberman:2013:CSM:2486788.2487002}, para poder
dinámicamente cargar un controlador en un ARDrone \cite{ARDrone}. La especificación, tanto vieja como nueva, son
simplificaciones del caso de estudio del Buscador UAV de vida salvaje (sección \ref{buscador_UAV}). 

El caso de estudio incluye las acciones de despegar (\emph{takeOff}), aterrizar ($land$), detectar marcas mediante la cámara
onboard ($read$) y objetivos relacionados con las acciones que el drone ejecuta, obteniendo feedback de los leds
onboard, cuando las marcas son leidas ($blink$). La diferencia entre especificación vieja y nueva es que al principio el
robot debe parpadear al leer una marca $x$ y luego de la actualización debe parpadear al leer una marca $x'$ distinta a
la original. Además, luego de la actualización tendremos un contador de batería que irá decrementando su valor a medida
que el ARDrone ejecute acciones. El nivel de batería no debe agotarse nunca para satisfacer los objetivos, y para esto
el robot puede ejecutar la acción $charge$ la que pondrá dicho contador en su nivel más alto.

Por cada una de estas acciones, que son parte de la especificación, definimos una clase $Action$ dentro de nuestro
enactment framework para ejecutar implementaciones del framework YaDrone \cite{YaDrone}.



