\section{Casos de Estudio}

\subsection*{Configuración experimental}

Todos los casos de estudios se ejecutan usando una extensión de la herramienta MTSA \cite{4639371} que nos da soporte
para especificar LTSs y propiedades, usando una notación textual, álgebra de procesos y FLTL. La herramienta también nos 
da soporte para síntesis de controladores para problemas de control SGR(1), los cuales son estrictamente mas caros de lo
que es requerido para problemas de control en la actualización de controladores dinámicamente. La herramienta fue 
extendida para automáticamente computar $A_u$, $G_u$ y $E_u$, y resuelve el problema de control de la actualización
dinámica de controladores. La versión extendida de la herramienta y la especificación completa de todos los casos de
estudio pueden verse en \cite{}.

\subsection{Planta de energía nuclear}

En \cite{PanzicaLaManna:2013:FCC:2487336.2487349} se analiza un controlador para un sistema de refrigeración de una
planta de energía nuclear. El controlador actual debe ejecutar servicios de mantenimiento primero apagando la
bomba que refrigera la planta y luego reiniciandola. El nuevo controlador no necesita parar la bomba y reiniciarla. Hay
también un invariante del sistema que indica que la bomba de refrigeración no puede estar apagada indefinidamente (esto
puede llevar a un accidente devastador). Los autores muestran que si una actualización se lleva a cabo en un estado en
donde el controlador actual tiene la bomba apagada, entonces el nuevo controlador puede no reiniciar la bomba, generando
un accidente. Esto nos muestra dos problemas. Primero, que la forma segura de preservar el invariante del sistema es
requerir que la actualización preserve el comportamiento como si el controlador actual haya sido reemplazado desde su
estado inicial (cuando la bomba es reiniciada). Esto es lo que los autores aseguran en \cite{6224401}. Segundo, que las
restricciones de cuándo una actualización de \cite{6224401} es suavisada como en \cite{PanzicaLaManna:2013:FCC:2487336.2487349} 
puede generar fallas y consecutivamente ellos recomiendan que ``el diseño de controladores actualizables
dinámicamente basados en el criterio débil de  actualización incluye una subsiguiente validación de los controladores''.

Nosotros resolvemos tres diferentes problemas de control de actualización dinámica de controladores para este caso de
estudio. El primero no tiene requerimientos para el período de transición entre la especificación vieja y la nueva (esto
es, $T$ configurado como $true$). El controlador obtenido ($C^1_u$) exhibe el comportamiento invalido descrito en
\cite{PanzicaLaManna:2013:FCC:2487336.2487349}, permitiendo una actualización mientras la bomba esta apagada, prendiendo
la bomba durante la transición y permitiendo al nuevo controlador a nunca reiniciarla.

El segundo problema de control usa a $T$ con el requerimiento genérico de \cite{6224401} (ver fórmula
\ref{ghezzi_formula}). El controlador resultante ($C^2_u$) evita la actualización mientras la bomba esta apagada, y por lo
tanto, evita que el nuevo controlador deje la bomba apagada sin intención. De hecho, como se espera, este controlador
exhibe estrictamente menos comportamiento que el controlador previo ($TR(C^2_u) \subset TR(C^1_u)$).




