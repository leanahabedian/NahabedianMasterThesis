\section{Resultados}

En esta sección detallaremos los resultados obtenidos para cada caso de estudio planteado en la sección
\ref{cases_of_study}. Para comparar diferentes resultados obtenidos por el problema de control notaremos $Tr(C)$ como el
conjunto de trazas del LTS $C$. Informalmente, una traza es una secuencia de acciones que un controlador puede
realizar. Por lo tanto, dado dos controladores $C_1$, $C_2$, si toda traza de $C_1$ está en $Tr(C_2)$ (i.e. $Tr(C_1) \subset
Tr(C_2)$) podremos decir que el controlador $C_2$ tiene mas comportamiento que $C_1$ debido a que en el controlador
$C_2$ puedo ejecutar todas las secuencias de acciones de $C_1$ y algunas más.

\subsection{Planta de energía nuclear}

Nosotros resolvemos tres diferentes problemas de control de actualización dinámica de controladores para este caso de
estudio. El primero no tiene requerimientos para el período de transición entre la especificación vieja y la nueva (esto
es, $T$ configurado como $true$). El controlador obtenido ($C^1_u$) exhibe el comportamiento invalido descrito en
\cite{PanzicaLaManna:2013:FCC:2487336.2487349}, permitiendo una actualización mientras la bomba esta apagada, prendiendo
la bomba durante la transición y permitiendo al nuevo controlador a nunca reiniciarla.

El segundo problema de control usa a $T$ con el requerimiento genérico de \cite{6224401} (ver fórmula
\ref{ghezzi_formula}). El controlador resultante ($C^2_u$) evita la actualización mientras la bomba esta apagada, y por
lo
tanto, evita que el nuevo controlador deje la bomba apagada sin intención. De hecho, como se espera, este controlador
exhibe estrictamente menos comportamiento que el controlador previo ($Tr(C^2_u) \subset Tr(C^1_u)$).

Finalmente, el tercer problema de control modela explícitamente el requerimiento de que la bomba no debería estar
apagada continuamente (esto es, en realidad, un requerimiento de transición de aceptación büchi que puede ser manejado
por nuestro enfoque sin problemas. Ver capítulo \ref{discusion}). El controlador resultante ($C^3_u$) no solo evita el
escenario descrito en \cite{PanzicaLaManna:2013:FCC:2487336.2487349} y exhibe menos comportamiento que el primer
controlador ($Tr(C^2_u) \subset Tr(C^1_u)$) sino que también provee más oportunidades de actualización que el segundo
controlador exhibiendo estrictamente más comportamiento ($Tr(C^2_u) \subset Tr(C^3_u)$). 

En otras palabras, nuestra técnica provee un criterio más relajado para controladores actualizables con respecto a
\cite{6224401} y que a su vez, es correcto por construcción ya que satisface los requerimientos de transición y además
no necesita de una validación manual posterior como en \cite{PanzicaLaManna:2013:FCC:2487336.2487349}.

