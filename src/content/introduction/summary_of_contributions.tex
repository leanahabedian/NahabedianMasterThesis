\section{Resumen de la contribución}

\begin{comment}
Trabajos previos en el área de ingeniería de los requerimientos han estudiado la utilidad de diversas técnicas de sistemas
adaptativos, debido a la necesidad de cambiar el ambiente o los requerimientos de un sistema sin frenar o interrumpir su
ejecución.

La contribución central de esta tesis es la presentación de una técnica que produce la actualización automática de
sistemas. Dicha técnica, propone mejorar ciertos problemas que las anteriores estrategias padecían, como la necesidad de
conocer estados de la ejecución en los cuales es ''seguro'' actualizar, y si lo es, detectaremos cual es el estado
equivalente en el sistema actualizado. Detallaremos en el capitulo BLA por qué y cómo resolvemos dichas complicaciones.

Por otro lado, lo novedoso del trabajo presentado es el enfoque usado. La actualización automática de sistemas ha sido
estudiado en diversos enfoques pero no bajo el marco de síntesis de controladores. Al presentar dicha técnica bajo un
marco no explorado abrimos a la comunidad científica un nuevo paradigma que podría potencialmente solucionar problemas
en otras áreas relacionadas con la ingeniería de los requerimientos.
\end{comment}

