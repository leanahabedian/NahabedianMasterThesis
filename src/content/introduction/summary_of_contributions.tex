\section{Resumen de la contribución}

En esta tesis, como en \cite{6224401,PanzicaLaManna:2013:FCC:2487336.2487349} consideramos el problema de controladores
actualizables en un sistema reactivo. Por otro lado, proponemos una técnica para actualización dinámica que
\emph{fuerza} al sistema a alcanzar un estado actualizable en vez de asumir que el sistema va a llegar a dicho estado.
Por lo tanto, nuestro enfoque no sólo garantiza que la nueva especificación va a valer \emph{si} la actualización se
produce, sino que también garantiza que la actualización \emph{va a suceder}. Además, generalizamos la noción de estados
actualizables en \cite{6224401,PanzicaLaManna:2013:FCC:2487336.2487349} mientras mantenemos correctitud, que es mejor que
requerir que la nueva especificación empiece a valer desde el estado inicial (o co-inicial 
\cite{PanzicaLaManna:2013:FCC:2487336.2487349}) del controlador actual. Proveemos también una especificación declarativa
y general para señala desde que punto los nuevos objetivos valen y designar propiedades que deben valer en el momento
que el sistema esta transicionando. 

Una clase de sistemas que se amoldan perfectamente al trabajo presentado son los sistemas adaptables. Dichos sistemas están diseñados
para que, mientras el proceso esta corriendo, pueda soportar cambios en cuanto a la disponibilidad de recursos o 
necesidades del usuario y también soportar condiciones inesperadas del ambiente y fallas \cite{SEAMS}. 

Nuestro enfoque de actualización de controladores dinámicos forma parte junto con otros trabajos de la misma área para
síntesis de controladores discretos (ej. \cite{21072}, \cite{Piterman},\cite{D'ippolito:2013:SNE:2430536.2430543}).
Síntesis de controladores automáticamente construye una estrategia operacional (en la forma de máquinas de estado) que
es capaz de garantizar un objetivo bajo asunciones del ambiente.

El problema de controladores dinámicos actualizables puede ser expresado como un problema de síntesis de controladores
en el cual el nuevo controlador cumple que:

\begin{itemize}
\itemsep-4mm
\item Es una estructura equivalente al controlador actual hasta que recibe un mensaje (no controlable) llamado
\emph{beginUpdate},
\item Satisface la especificación actual hasta que el evento (controlable) \emph{stopOldSpec} sucede,
\item Garantiza a la nueva especificación desde que ocurre la acción (controlada) \emph{startNewSpec},
\item Proporciona que el comportamiento entre \emph{beginUpdate}, \emph{stopOldSpec} y \emph{startNewSpec} satisface
cualquier requerimiento de transición y
\item Garantiza que los eventos \emph{startNewSpec} y \emph{stopOldSpec} van a suceder eventualmente.
\end{itemize}


%Trabajos previos en el área de ingeniería de los requerimientos han estudiado la utilidad de diversas técnicas de sistemas
%adaptativos, debido a la necesidad de cambiar el ambiente o los requerimientos de un sistema sin frenar o interrumpir su
%ejecución.

%La contribución central de esta tesis es la presentación de una técnica que produce la actualización automática de
%sistemas. Dicha técnica, propone mejorar ciertos problemas que las anteriores estrategias padecían, como la necesidad de
%conocer estados de la ejecución en los cuales es ''seguro'' actualizar, y si lo es, detectaremos cual es el estado
%equivalente en el sistema actualizado. Detallaremos en el capitulo BLA por qué y cómo resolvemos dichas complicaciones.

%Por otro lado, lo novedoso del trabajo presentado es el enfoque usado. La actualización automática de sistemas ha sido
%estudiado en diversos enfoques pero no bajo el marco de síntesis de controladores. Al presentar dicha técnica bajo un
%marco no explorado abrimos a la comunidad científica un nuevo paradigma que podría potencialmente solucionar problemas
%en otras áreas relacionadas con la ingeniería de los requerimientos.

