\section{Motivación}

Cuando trabajamos en ingeniería de requerimientos, nuestra tarea es relevar y documentar los objetivos y el
funcionamiento del ambiente, para elaborar un conjunto de requerimientos cuya máquina a construir debe cumplir.
Teniendo estas tres componentes, podemos formular la siguiente ecuación: R, D $\vDash$ G. Donde R son los Requerimientos de la
máquina, D las asunciones del dominio y G los objetivos que la máquina deberá satisfacer.

Entonces, un problema clave de la ingeniería de requerimientos puede ser visto como un problema de síntesis, tal que
dado un modelo de la máquina, un modelo del mundo y un conjunto de objetivos del sistema, permite construir un modelo
operacional el cual satisface los objetivos. La técnica que resuelve esta ecuación mediante los modelos mencionados es
llamada síntesis de controladores y esta siendo estudiada exhaustivamente en varios aspectos de la ingeniería de los requerimientos.

La construcción de dichos modelos son una de las principales herramientas en el diseño de sistemas concurrentes, debido
a que nos facilita la detección de errores de diseño en las primeras etapas de desarrollo. Por otro lado, estos modelos
de comportamiento pueden resultar complejos de construir. Utilizar la técnica de síntesis anteriormente mencionada
facilita la construcción tomando modelos pequeños y propiedades lógicas, que suelen ser más sencillas de especificar y de validar.

Por lo tanto, un problema de control consiste en generar automáticamente una máquina que restrinja la ocurrencia de
eventos controlables basado en los eventos del mundo que se producen. Es decir, teniendo la especificación de un
ambiente, asunciones, objetivos y un conjunto de acciones controlables, podemos definir una máquina cuyo comportamiento
concurrente con el ambiente satisfaga las asunciones y los objetivos del sistema.

Este problema está siendo ampliamente estudiado en diferentes contextos, desde sistemas autoadaptativos hasta
composición automática de web-services. Para diseñar softwares autoadaptativos los más evolucionados fueron los basados
en arquitectura y los basados en requerimientos. El primero de estos utiliza reglas de adaptación que serán utilizadas
para monitorear condiciones del conjunto de operaciones del sistema y definir acciones en run-time si las condiciones
son desfavorables. Por otro lado, los basados en requerimientos extienden las técnicas de ingeniería de los
requerimientos con el fin de representar a los requisitos de adaptación y/o la incertidumbre inherente del medio
ambiente en el que opera el sistema. El modelo más utilizado en la comunidad, hoy en día, para definir sistemas
autoadaptables es el modelo arquitectural. 

Actualmente, trabajos recientes en el área de sistemas auto adaptativos definen una técnica para lograr encontrar
estados de la ejecución del controlador en los cuales es seguro realizar una actualización y seguir  ejecutando el nuevo
controlador sin tener que apagar y prender el primero. Sin embargo, dicho problema no ha sido estudiado en el marco de
síntesis de controladores.

