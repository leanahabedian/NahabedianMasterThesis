\section{Motivación}

La operación continua, sistemas donde cada una de sus componentes se mantienen operativos todo el tiempo, es un
requerimiento común en muchas aplicaciones. Por lo tanto, es necesario desarrollar técnicas ingenieriles que puedan
actualizar un sistema, tanto su ambiente como sus requerimientos, sin la necesidad de frenar o interrumpir sus
operaciones. Este trabajo ha sido estudiado de diversas maneras, empezando por la actualización dinámica de software
\cite{60317} y más recientemente con el diseño de software adaptable \cite{SEAMS}.

La pregunta central que intenta solucionar este problema es ?`cuándo es seguro cambiar un componente de software en un sistema
que esta corriendo? Una respuesta conservadora a esta pregunta es ``cuando los componentes no están involucrados en
alguna interacción''; esto fue formalizado introduciendo la noción de \textbf{quietud} (\emph{quiescense}) \cite{60317} y luego
\textbf{tranquilidad} (\emph{tranquility}) \cite{4359466}. Muchas otras técnicas han sido desarrolladas (como en
\cite{Anderson:2009:MPM:1656437.1656448} y \cite{485222}) aunque estas nunca explican los requerimientos de
actualización \cite{Baresi:2010:DBD:1882362.1882367}, ni indican cuando es correcto realizar el cambio a la nueva especificación.
Para tal fin Ghezzi et al. \cite{6224401,PanzicaLaManna:2013:FCC:2487336.2487349} estudiaron el problema de actualizar un
controlador que esta monitoreando un ambiente de sistema reactivo mientras controla actuadores. La pregunta que persiguen
contestar los autores es ?`cuándo es seguro reemplazar el controlador actual con uno nuevo, donde se cumple la nueva
especificación?

Un problema en común que los trabajos existentes en actualización dinámica poseen, es que dichas técnicas necesitan
asumir que el sistema que esta siendo ejecutado va a eventualmente alcanzar un estado seguro donde hacer la
actualización. Estos estados son designados como \emph{actualizables} y su identificación depende de cada técnica.
Dichas técnicas, suelen tener un operador o una pieza de software especial designado a identificar dichos estados. Esta asunción, es algo
que no depende del software, sino que depende del ambiente, el cual sabemos que no puede ser manipulado. A su vez, los trabajos
existentes, tampoco dan una técnica para guiar al sistema hacia un estado actualizable. 

Por ejemplo, en \cite{60317}, la expectativa sería que los componentes en un sistema distribuido deben ser diseñados, para que
den información del momento en que dicho componente entró en estado de \emph{quiescence}. A su vez, están diseñados para aceptar mensajes
\emph{pasivate} que, al deshabilitar componentes para inicializar nuevas transacciones, intentan, pero no garantizan
que alcance \emph{quiescence}. Similarmente, en \cite{6224401} requiere asumir que el controlador a ser actualizado va a
volver, eventualmente, a su estado inicial (o asumir que la nueva especificación vale desde el último estado inicial). Estados
actualizables son aquellos en los que el comportamiento del sistema desde el ultimo estado inicial puede controlarse
para satisfacer la nueva especificación.  Por otra parte, en \cite{PanzicaLaManna:2013:FCC:2487336.2487349}, el mismo
autor relaja las condiciones necesarias para realizar una actualización, permitiendo más estados actualizables, al
costo de violar la nueva especificación, y sin contemplar el problema de garantizar que dicho estado sea alcanzado.

%Cuando trabajamos en ingeniería de requerimientos, nuestra tarea es relevar y documentar los objetivos y el
%funcionamiento del ambiente, para elaborar un conjunto de requerimientos cuya máquina a construir debe cumplir.
%Teniendo estas tres componentes, podemos formular la siguiente ecuación: R, D $\vDash$ G. Donde R son los Requerimientos de la
%máquina, D las asunciones del dominio y G los objetivos que la máquina deberá satisfacer.

%Entonces, un problema clave de la ingeniería de requerimientos puede ser visto como un problema de síntesis, tal que
%dado un modelo de la máquina, un modelo del mundo y un conjunto de objetivos del sistema, permite construir un modelo
%operacional el cual satisface los objetivos. La técnica que resuelve esta ecuación mediante los modelos mencionados es
%llamada síntesis de controladores y esta siendo estudiada exhaustivamente en varios aspectos de la ingeniería de los requerimientos.

%La construcción de dichos modelos son una de las principales herramientas en el diseño de sistemas concurrentes, debido
%a que nos facilita la detección de errores de diseño en las primeras etapas de desarrollo. Por otro lado, estos modelos
%de comportamiento pueden resultar complejos de construir. Utilizar la técnica de síntesis anteriormente mencionada
%facilita la construcción tomando modelos pequeños y propiedades lógicas, que suelen ser más sencillas de especificar y de validar.

%Por lo tanto, un problema de control consiste en generar automáticamente una máquina que restrinja la ocurrencia de
%eventos controlables basado en los eventos del mundo que se producen. Es decir, teniendo la especificación de un
%ambiente, asunciones, objetivos y un conjunto de acciones controlables, podemos definir una máquina cuyo comportamiento
%concurrente con el ambiente satisfaga las asunciones y los objetivos del sistema.

%Este problema está siendo ampliamente estudiado en diferentes contextos, desde sistemas autoadaptativos hasta
%composición automática de web-services. Para diseñar softwares autoadaptativos los más evolucionados fueron los basados
%en arquitectura y los basados en requerimientos. El primero de estos utiliza reglas de adaptación que serán utilizadas
%para monitorear condiciones del conjunto de operaciones del sistema y definir acciones en run-time si las condiciones
%son desfavorables. Por otro lado, los basados en requerimientos extienden las técnicas de ingeniería de los
%requerimientos con el fin de representar a los requisitos de adaptación y/o la incertidumbre inherente del medio
%ambiente en el que opera el sistema. El modelo más utilizado en la comunidad, hoy en día, para definir sistemas
%autoadaptables es el modelo arquitectural. 

%Actualmente, trabajos recientes en el área de sistemas auto adaptativos definen una técnica para lograr encontrar
%estados de la ejecución del controlador en los cuales es seguro realizar una actualización y seguir  ejecutando el nuevo
%controlador sin tener que apagar y prender el primero. Sin embargo, dicho problema no ha sido estudiado en el marco de
%síntesis de controladores.
