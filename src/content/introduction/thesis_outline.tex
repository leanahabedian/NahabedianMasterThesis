\section{Esquema de tesis}

Este trabajo está organizado de la siguiente manera. Empezaremos dando una base de conocimientos teóricos en el capítulo
\ref{background}. Luego de esto, en el capítulo \ref{MTSA_tool} vamos a explicar el soporte que nos da la herramienta
MTSA para poder manejar y controlar todos los conceptos que fueron definidos en el capítulo previo, como el modelado y
la síntesis de controladores.

La técnica central de este trabajo y el aporte científico estará principalmente desarrollado en el capítulo
\ref{updating_controller_chapter}. En éste detallaremos la aplicación de la técnica de síntesis de controladores al
problema de actualización dinámica. Explicaremos además, cómo construimos la solución propuesta y daremos las
definiciones necesarias para finalmente demostrar la correctitud de nuestro enfoque.

En el capítulo \ref{validation}, realizamos un trabajo comparativo entre las técnicas que existen actualmente y la propuesta por
nosotros. Los resultados obtenidos destacando ventajas y desventajas de cada estrategia son expuestos. Explicaremos
además cual es la motivación de utilizar cada caso de estudio.

Finalmente, detallamos los límites y alcances de nuestro trabajo en el capítulo \ref{discusion}. Discutiremos otros
enfoques comparando nuestra labor con trabajos ya existentes. Resaltaremos posibles trabajos futuros que pueden surgir a
partir de este trabajo y concluiremos con un resumen.
